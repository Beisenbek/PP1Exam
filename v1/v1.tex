\documentclass[11pt]{article}

\usepackage{amsmath,amssymb,amsfonts}
\usepackage{graphicx}
\setlength{\topmargin}{-.5in} \setlength{\textheight}{9.25in}
\setlength{\oddsidemargin}{0in} \setlength{\textwidth}{6.8in}
\graphicspath{ {../images/} }
\usepackage{listings}
\usepackage{xcolor}
% set the default code style
\lstset{
    frame=tb, % draw a frame at the top and bottom of the code block
    tabsize=4, % tab space width
    showstringspaces=false, % don't mark spaces in strings
    commentstyle=\color{green}, % comment color
    keywordstyle=\color{blue}, % keyword color
    stringstyle=\color{red} % string color
}

\begin{document}
\Large

\begin{center}
\includegraphics[width=9cm, height=2cm]{kbtu.jpg}
\end{center}

\begin{center}
	\begin{minipage}{11.4cm}
		\begin{center}
				{\small \textsc{Kazakh-British Technical University}			\\
						  \textsc{Faculty of Information technologies} \\
                         \textbf{Course:} Programming Principles I \hspace{.65cm}\textbf{Date:} 14/12/2019\\\textbf{Variant:} 1\\
                }
		\end{center}
	\end{minipage}
\end{center}

{If you have multiple answers for one question fill them in one cell}
\begin{center} 
    \begin{tabular}{ |c|c|c|c|c|c|c|c|c|c|c|}
 \hline
 01 & 02 & 03 & 04 & 05 & 06 & 07 & 08 & 09 & 10 \\ 
  \hline
   &&&&&&&&& \\ 
 \hline
  11 & 12 & 13 & 14 & 15 & 16 & 17 & 18 & 19 & 20\\ 
  \hline
  &&&&&&&&& \\ 
 \hline
\end{tabular}
\end{center}

\medskip\hrule
\begin{enumerate}

\item Which of the following data types takes 1 byte of memory?
\begin{enumerate}
    \item int
    \item long long
    \item char +
    \item double
\end{enumerate}

\item Logical operator XOR is denoted by:
\begin{enumerate}
    \item \%
    \item \#
    \item \textsuperscript{$\wedge$} +
    \item |
\end{enumerate}


\item What is the output of following program?
\begin{lstlisting}[language=C++]
#include <iostream>
using namespace std;
int main(){
    int x = 15, y = 20;
    x += 23;
    y++;
    cout << x % y;
    return 0;
}

\end{lstlisting}
\begin{enumerate}
    \item 19
    \item 17+
    \item 3
    \item None of them
\end{enumerate}



\end{enumerate}

\end{document} 