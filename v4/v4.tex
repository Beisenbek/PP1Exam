\documentclass[10pt]{article}

\renewcommand{\seriesdefault}{\bfdefault}

\usepackage{amsmath,amssymb,amsfonts}
\usepackage{graphicx}
\setlength{\topmargin}{0in} \setlength{\textheight}{9.25in}
\setlength{\oddsidemargin}{0in} \setlength{\textwidth}{6.8in}
\graphicspath{ {../images/} }
\usepackage{listings}
\usepackage{xcolor}
\usepackage{multicol}
% set the default code style
\lstset{
    frame=tb, % draw a frame at the top and bottom of the code block
    tabsize=2, % tab space width
    showstringspaces=false, % don't mark spaces in strings
    commentstyle=\color{green}, % comment color
    keywordstyle=\color{blue}, % keyword color
    stringstyle=\color{orange} % string color
}

\begin{document}

\begin{center}
\includegraphics[width=8cm, height=2cm]{kbtu.jpg}
\end{center}

\begin{center}
	\begin{minipage}{11.4cm}
		\begin{center}
				{\small \textsc{Kazakh-British Technical University}			\\
						  \textsc{Faculty of Information technologies} \\
                         \textbf{Course:} Programming Principles I \hspace{.65cm}
                         \textbf{Date:} 14/12/2019\\\textbf{Variant:} 4\\
                }
		\end{center}
	\end{minipage}
\end{center}
\textbf{
{Full name:}\\
{Answers section}\\
}
{If you have multiple answers for one question write them all}
\begin{multicols}{5}
\begin{enumerate}
\item \item \item \item \item \item \item \item \item \item \item \item \item \item \item \item \item \item \item \item
\end{enumerate}
\end{multicols}
\medskip\hrule
\textbf{
\\
{Questions section}\\
}
\begin{enumerate}

\item Which of the data types can not store “123,678” value?
\begin{multicols}{4}
\begin{enumerate}
    \item string
    \item float
    \item int
    \item double
\end{enumerate}
\end{multicols}




\item What will be the output of this code:
\begin{lstlisting}[language=C++]
#include <iostream>
using namespace std;
int main(){
    int n = 135;
    int k =  n%10;
    if(n%2==1 && n%k==0)
        cout << "yes";
    else
        cout << "no";
    return 0;
}

\end{lstlisting}
\begin{multicols}{4}
\begin{enumerate}
    \item yes
    \item no
    \item nan
    \item none of the above
\end{enumerate}
\end{multicols}





\item What will be the output of this code:
\begin{lstlisting}[language=C++]
#include <iostream>
using namespace std;
int main()
{
    int a[5] = {1,2,3,4,5};
    for (int i=4;i>=0;i--)
    {
        cout<<a[i]<<" ";
    }
}

\end{lstlisting}
\begin{multicols}{4}
\begin{enumerate}
    \item 1 2 3 4 5
    \item 5 4 3 2 1
    \item 4 3 2 1
    \item 4 3 2 1 5
\end{enumerate}
\end{multicols}










\item What will be the output of this code:
\begin{lstlisting}[language=C++]
#include <iostream>
using namespace std;
int main(){
    string s;
    cin >> s;
    cout << s << endl;
    return 0;
}
\end{lstlisting}
\begin{multicols}{4}
\begin{enumerate}
    \item Hello
    \item Hello world
    \item world
    \item none of the above
\end{enumerate}
\end{multicols}








\item What is wrong in the following code:
\begin{lstlisting}[language=C++]
#include <iostream>
using namespace std;
int main()
{
    int n = 3;
    double score[n];
    for(int i=0;i<=n;i++)
    {
        cout<<"Enter score\n";
        cin>>score[i];
    }
    return 0;
}
\end{lstlisting}
\begin{multicols}{4}
\begin{enumerate}
    \item Array must be ‘int’ type
    \item It should be ‘cin‘ for score[n];
    \item ‘i’ must be strictly less than ‘n’
    \item Everything is right
\end{enumerate}
\end{multicols}




\item What is return type of function with no return value
\begin{multicols}{4}
\begin{enumerate}
    \item int
    \item char
    \item long long
    \item void
\end{enumerate}
\end{multicols}





\item What is the output of the following code
\begin{lstlisting}[language=C++]
#include <iostream>
#include <vector>
using namespace std;
int main()
{
    vector<int> v;
    v.push_back(10);
    v.push_back(20);
    v.push_back(30);
    v.push_back(40);
    v.push_back(50);
    int erase = 3;
    v.erase(v.begin()+erase-1);
    vector<int>::iterator it;
    for(it = v.begin();it!=v.end();it++)
    {
        cout<<*it<<" ";
    }
    return 0;
}
\end{lstlisting}
\begin{multicols}{4}
\begin{enumerate}
    \item 10 20
    \item 10 20 30 50
    \item 10 20 40 50
    \item none of the above
\end{enumerate}
\end{multicols}






\item How set will look if we add this array elements to it? 
\begin{lstlisting}[language=C++]
string arr[] = {"Programming", "KBTU", 
"kbtu", "19BD", "Calculus", "KBTU"};
\end{lstlisting}
\begin{enumerate}
    \item Programming, KBTU, kbtu, 19BD, Calculus, KBTU
    \item 19BD, Calculus, KBTU, Programming, kbtu
    \item 19BD, Calculus, KBTU, Programming
    \item Calculus, KBTU, Programming, kbtu, 19BD
\end{enumerate}





\item You have map with these values:
\begin{lstlisting}[language=C++]
map.insert(pair<string, int>("one", 10));
map.insert(pair<string, int>("two", 20));
map.insert(pair<string, int>("three", 60));
map.insert(pair<string, int>("four", 40));
map.insert(pair<string, int>("five", 80));
map.insert(pair<string, int>("six", 10));
map.insert(pair<string, int>("seven", 60));
\end{lstlisting}
What code
\begin{lstlisting}[language=C++]
cout<<map["three"]<<endl; 
\end{lstlisting}
will do ?
\begin{enumerate}
    \item output "three"
    \item output "60"
    \item output "three 60"
    \item nothing, we can not cout map without iterator
\end{enumerate}



\item Calculate the following expression (true AND false) XOR (true OR false) 
\begin{enumerate}
    \item true
    \item false
    \item impossible to calculate
    \item 1.3
\end{enumerate}



\item What is the output of following program?
\begin{lstlisting}[language=C++]
#include <iostream>
using namespace std;
int main()
{
   int arr[10] = {1, 2, 3, 4, 5, 6, 7, 8, 9, 10};
       for(int i = 1 ; i <= 10; i++)
       {
           if(arr[i] % 2 != 0 and i % 2 == 0)
               cout << i << " ";
       }
 
   return 0;
}
\end{lstlisting}
\begin{enumerate}
    \item 2 4 6 8
    \item 3 5 7 9
    \item 2 4 6 7
    \item 1 3 5 7
\end{enumerate}


\item What is the output of following program?
\begin{lstlisting}[language=C++]
#include <iostream>
using namespace std;
int main()
{
   string s1 = "Ali";
   string s2 = "Alik";
   string s3 = s 1+ s2;
 
       for(int i = 1 ; i < s3.length() ; i++)
       {
           if(i % 2 == 0)
           {
               cout << s3[i] << "-";
           }
       }
  
   return 0;
}
\end{lstlisting}
\begin{enumerate}
    \item i-l-k
    \item A-l-i-k-
    \item i-l-k-
    \item i-i-l-k-
\end{enumerate}


\item What is the output of following program?
\begin{lstlisting}[language=C++]
#include <iostream>
using namespace std;
string check(int a)
{
   if(a % 2 != 0)
       return "yes";
   else
       return "No";
}
int main()
{
    int arr[8] = {122,32,41,43,56,53,67,77};
 
    for(int i = 0 ; i <=7 ; i++)
        cout << check(arr[i]) << " ";

   return 0;
}

\end{lstlisting}
\begin{enumerate}
    \item no no yes yes no yes yes yes
    \item No No yes yes No yes yes yes
    \item No No yes yes No yes No yes
    \item No No yes yes yes yes yes yes
\end{enumerate}



\item What is the output of following program?
\begin{lstlisting}[language=C++]
#include <iostream>
#include <vector>
using namespace std;
int main()
{
   vector<int> v;
   vector<int> v1;
   v.push_back(1);
   v1.push_back(3);
   v1.push_back(5);
   v.push_back(01);
   v.pop_back();
   v1.push_back(2);
   v.push_back(4);
   v1.push_back(2);
   v1.pop_back();
   v.pop_back();
   cout << v1.front() << " " << v.back() << " " << v.front();
   return 0;
}

\end{lstlisting}
\begin{enumerate}
    \item 3 1 1
    \item 3 01 1
    \item 1 1 3
    \item 3 2 1
\end{enumerate}




\item What is the output of following program?
\begin{lstlisting}[language=C++]
#include <iostream>
#include <iostream>
#include <map>
using namespace std;
int main()
{
    int a[10] = {1, 3, 4, 2, 1, 7, 4, 5, 6, 7};
    map<int, int> m;
    for(int i = 3 ; i < 10 ; i++){
        if(i % 2 == 0)
        m[i] += a[i];
        else m[i+1] -= a[i];
    }
    map<int,int>::iterator it;
      
    for( it = m.begin() ; it!=m.end() ; it++)
        cout << (*it).first << " " << (*it).second << " ";
    
    return 0;
}
\end{lstlisting}
\begin{enumerate}
    \item 4 -1 6 -3 8 1 10 2
    \item -1 6 -3 8 1 10 3
    \item 4 -1 6 -3 8 1 10 -7
    \item -1 6 -3 8 1 10 -7
\end{enumerate}








\item What is the output of following program?
\begin{lstlisting}[language=C++]
#include <iostream>
#include <set>
using namespace std;
int main()
{
    int a[10] = {1, 1, 2, 3, 4, 5, 3, 2, 13, 3};
    set <int> s;
    for(int i = 3 ; i < 10 ; ++i)
    {
        s.insert(a[i]);
    }
 
    set <int>::iterator it;
 
    for(it = s.begin() ; it != s.end() ; it++)
       cout << *it << " ";
   return 0;
}
\end{lstlisting}
\begin{enumerate}
    \item 2 3 4 5 13
    \item 1 2 3 4 5 13
    \item 2 3 4 5 13
    \item 3 4 5 13 14
\end{enumerate}







\item What is the output of following program?
\begin{lstlisting}[language=C++]
#include <iostream>
using namespace std;
int f(int a,int b)
{
   return a|b;
}
int main()
{
   int a[4] = {1,2,3,4};
 
       for(int i = 0 ; i < 4 ; i=i+2)
           cout<<f(a[i],a[i+1])<<" ";
  
   return 0;
}
\end{lstlisting}
\begin{enumerate}
    \item 3 8
    \item 3 7
    \item 2 7
    \item 1 6
\end{enumerate}




\item What is the output of following program?
\begin{lstlisting}[language=C++]
#include <iostream>
#include <cmath>
using namespace std;
bool check(int n)
{
   for(int i = 2 ; i <= sqrt(n) ; i++)
   {
       if(n % i == 0)
           return 0;
   }
 
   return 1;
}
int main()
{
   int n = 765423;
   int res = 0;
       while(n > 0)
       {
           int x = n % 10;
           if(check(x)) res += x;
           else res -= x;
           n /= 10;
       }
       cout << res;
   return 0;
}

\end{lstlisting}
\begin{enumerate}
    \item 5
    \item 7
    \item 1
    \item 0
\end{enumerate}




\item What is the output of following program?
\begin{lstlisting}[language=C++]
#include <iostream>
 
using namespace std;
 
int main() {
    int n = 10;
    while(n > 0) {
        if (n & 1)
            cout << n << " ";
        n--;
    }
    return 0;
}


\end{lstlisting}
\begin{enumerate}
    \item 10 9 8 7 6 5 4 3 2 1
    \item 9 7 5 3 1
    \item 10 8 6 4 2 
    \item No output
\end{enumerate}



\item What is the output of following program?
\begin{lstlisting}[language=C++]
#include <iostream>
using namespace std;
int main() {
    int cnt = 1;
    for (int i = 0; i < 5; i++) {
        int k = 0;
        for (int j = 0; j < 2; j++) {
            k++;
        }
        cnt*=k;
    }
    cout << cnt;
    return 0;
}
\end{lstlisting}
\begin{enumerate}
    \item 16
    \item 32
    \item 64 
    \item 10
\end{enumerate}


\end{enumerate}
\textbf{
Minutes \#2 of Faculty of Information Technology meeting November 11, 2019 \\\\
Teachers: Baisakov B. M., Akshabayev A. K. \\\\ 
Dean of FIT Suliev R. N \\\\
}
\end{document} 