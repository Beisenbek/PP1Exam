\documentclass[10pt]{article}

\usepackage{amsmath,amssymb,amsfonts}
\usepackage{graphicx}
\setlength{\topmargin}{0in} \setlength{\textheight}{9.25in}
\setlength{\oddsidemargin}{0in} \setlength{\textwidth}{6.8in}
\graphicspath{ {../images/} }
\usepackage{listings}
\usepackage{xcolor}
\usepackage{multicol}
% set the default code style
\lstset{
    frame=tb, % draw a frame at the top and bottom of the code block
    tabsize=2, % tab space width
    showstringspaces=false, % don't mark spaces in strings
    commentstyle=\color{green}, % comment color
    keywordstyle=\color{blue}, % keyword color
    stringstyle=\color{orange} % string color
}

\begin{document}

\begin{center}
\includegraphics[width=8cm, height=2cm]{kbtu.jpg}
\end{center}

\begin{center}
	\begin{minipage}{11.4cm}
		\begin{center}
				{\small \textsc{Kazakh-British Technical University}			\\
						  \textsc{Faculty of Information technologies} \\
                         \textbf{Course:} Programming Principles I \hspace{.65cm}
                         \textbf{Date:} 14/12/2019\\\textbf{Variant:} 3\\
                }
		\end{center}
	\end{minipage}
\end{center}
\textbf{
{Answers section}\\
}
{If you have multiple answers for one question write them all}
\begin{multicols}{5}
\begin{enumerate}
\item \item \item \item \item \item \item \item \item \item \item \item \item \item \item \item \item \item \item \item
\end{enumerate}
\end{multicols}
\medskip\hrule
\textbf{
\\
{Questions section}\\
}
\begin{enumerate}


\item What will be the output of this code:
\begin{lstlisting}[language=C++]
#include <iostream>
using namespace std;
int main(){
    for(int i = 0; i <= 5; i++) {
        if (i % 2 == 0)
            continue;
       else
            cout << i << " ";
   	}
    return 0;
}
\end{lstlisting}
\begin{multicols}{4}
\begin{enumerate}
    \item 0 2 4 
    \item 0 1 2 3 4 5 
    \item 1 3
    \item 1 3 5 
\end{enumerate}
\end{multicols}











\item What will be the output of this code:
\begin{lstlisting}[language=C++]
#include <iostream>
#include <algorithm>
using namespace std;
int main(){
    int a[4] = {20, 4, -1, -8};
    sort(a, a + 4);
    for (int i = 0; i < 3; i++)
        cout << a[i] << " ";
    return 0;
}
\end{lstlisting}
\begin{multicols}{4}
\begin{enumerate}
    \item 20 4 -1 -8 
    \item  -8 -1 4
    \item 4 -1 -8
    \item -8 -1 4 20 
\end{enumerate}
\end{multicols}








\item What will be the output of this code:
\begin{lstlisting}[language=C++]
using namespace std;
int main(){
    set<int> s;
    s.insert(5);
    s.insert(6);
    s.insert(7);
    s.insert(5);
    set<int> :: iterator it;
    for (it = s.begin(); it != s.end(); it++)
        cout << *it << " ";
    return 0;
}
\end{lstlisting}
\begin{multicols}{4}
\begin{enumerate}
    \item None of them 
    \item 7 6 5
    \item 5 5 6 7
    \item 5 6 7 
\end{enumerate}
\end{multicols}










\item What will be the output of this code:
\begin{lstlisting}[language=C++]
#include <iostream>
using namespace std;
int main(){
    string s = "CheAtiNg";
    for (int i = 0; i < s.size();i++) {
        if (s[i] >= 'A' && s[i] <= 'Z')
            cout << s[i + 1];
    }
    return 0;
}
\end{lstlisting}
\begin{multicols}{4}
\begin{enumerate}
    \item h t g
    \item c a n
    \item h g
    \item g t h 
\end{enumerate}
\end{multicols}







\item What will be the output of this code:
\begin{lstlisting}[language=C++]
#include <iostream>

using namespace std;

int recursion(int n) {
    if (n == 1) {
        return 1;
    }
    return recursion(n - 2) * n;
}

int main()
{
    int n = 7;
    cout << recursion(n);
    return 0;
}

\end{lstlisting}
\begin{multicols}{4}
\begin{enumerate}
    \item 5040
    \item 720
    \item 105
    \item 28
\end{enumerate}
\end{multicols}








\item Evaluate the next expression, where n = 135:
\begin{lstlisting}[language=C++]
n << 0
\end{lstlisting}
\begin{multicols}{4}
\begin{enumerate}
    \item 270
    \item 67
    \item 1350
    \item None of the above
\end{enumerate}
\end{multicols}








\item What will be the output of this code:
\begin{lstlisting}[language=C++]
#include <iostream>
#include<vector>

using namespace std;

int main(){
    vector<int> v(5);
    v.push_back(0);
    v.push_back(7);
    v.push_back(0);
    v.push_back(10);
    v.push_back(2);
    
    for (int i = 0; i < v.size() - 1; i++) {
        bool ok = (v[i] & v[i+1]);
        if (ok)
            cout << "ok" << " ";
    }

    return 0;

}


\end{lstlisting}
\begin{multicols}{4}
\begin{enumerate}
    \item ok
    \item ok ok
    \item ok ok ok
    \item None of the above
\end{enumerate}
\end{multicols}









\item What will be the output of this code:
\begin{lstlisting}[language=C++]
#include <iostream> 
#include <map> 
   
using namespace std; 
   
int main() { 
     map<int, int> marks; 
     marks.insert(pair<int, int>(160, 42)); 
     marks.insert(pair<int, int>(166, 34)); 
   
     map<int, int>::iterator it; 

     for (it =  marks.begin(); it !=  marks.end(); ++it) { 
        cout << *it << " "; 
     } 
     return 0;     
  }

\end{lstlisting}
\begin{multicols}{4}
\begin{enumerate}
    \item 160 42 166 34
    \item 160 166
    \item 42 34
    \item None of the above
\end{enumerate}
\end{multicols}






\item What will be the output of this code:
\begin{lstlisting}[language=C++]
#include<iostream>

using namespace std;
void swap(int m, int n) {
    int x = m;
    m = n;
    n = x;
}
main() {
    int x = 5, y = 3;
    swap(x, y);
    cout << x << " " << y;
    return 0;
}
\end{lstlisting}
\begin{multicols}{4}
\begin{enumerate}
    \item 3 5
    \item 5 3
    \item 5 5
    \item Compile error
\end{enumerate}
\end{multicols}





\item What will be the output of this code:
\begin{lstlisting}[language=C++]
#include<iostream>
using namespace std;
main() { 
   int x = 5;
   if(x == 5) {	
      if(x == 5) break;
      cout << "Hello";
   } 
  cout << "Hi"; 
}
return 0;

\end{lstlisting}
\begin{multicols}{4}
\begin{enumerate}
    \item Compile error
    \item Hi
    \item HelloHi
    \item Hello
\end{enumerate}
\end{multicols}






\end{enumerate}
\textbf{
Minutes \#2 of Faculty of Information Technology meeting November 11, 2019 \\\\
Teachers: Baisakov B. M., Akshabayev A. K. \\\\ 
Dean of FIT Suliev R. N \\\\
}
\end{document} 