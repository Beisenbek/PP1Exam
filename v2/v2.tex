\documentclass[10pt]{article}

\usepackage{amsmath,amssymb,amsfonts}
\usepackage{graphicx}
\setlength{\topmargin}{0in} \setlength{\textheight}{9.25in}
\setlength{\oddsidemargin}{0in} \setlength{\textwidth}{6.8in}
\graphicspath{ {../images/} }
\usepackage{listings}
\usepackage{xcolor}
\usepackage{multicol}
% set the default code style
\lstset{
    frame=tb, % draw a frame at the top and bottom of the code block
    tabsize=2, % tab space width
    showstringspaces=false, % don't mark spaces in strings
    commentstyle=\color{green}, % comment color
    keywordstyle=\color{blue}, % keyword color
    stringstyle=\color{orange} % string color
}

\begin{document}

\begin{center}
\includegraphics[width=8cm, height=2cm]{kbtu.jpg}
\end{center}

\begin{center}
	\begin{minipage}{11.4cm}
		\begin{center}
				{\small \textsc{Kazakh-British Technical University}			\\
						  \textsc{Faculty of Information technologies} \\
                         \textbf{Course:} Programming Principles I \hspace{.65cm}
                         \textbf{Date:} 14/12/2019\\\textbf{Variant:} 2\\
                }
		\end{center}
	\end{minipage}
\end{center}
\textbf{
{Answers section}\\
}
{If you have multiple answers for one question write them all}
\begin{multicols}{5}
\begin{enumerate}
\item \item \item \item \item \item \item \item \item \item \item \item \item \item \item \item \item \item \item \item
\end{enumerate}
\end{multicols}
\medskip\hrule
\textbf{
\\
{Questions section}\\
}
\begin{enumerate}


\item Which of the following data types CANNOT store the number 47255?
\begin{multicols}{4}
\begin{enumerate}
    \item int
    \item float
    \item char 
    \item double
\end{enumerate}
\end{multicols}



\item What will be the output of this code, if variable "in" will be 0.
\begin{lstlisting}[language=C++]
#include <iostream>
using namespace std;
int main(){
    int in;
    string num[10] = {"Greater than 9", "one", "two", "three", "four",
                      "five", "six", "seven", "eight", "nine"};
    cin >> in;
    if(in > 9){
        cout << num[0];
    }
    else{
            cout << num[in];
    }
    return 0;
}
\end{lstlisting}
\begin{multicols}{4}
\begin{enumerate}
    \item One
    \item Two
    \item It will stop with and error 
    \item Greater than 9 
\end{enumerate}
\end{multicols}





\item What will be the output of this code:
\begin{lstlisting}[language=C++]
#include <iostream>
using namespace std;
int main(){
    int i = 0;
    if (i == '0') cout << "Cheating is bad";
    else if (i == '1') cout << "Don't Cheat";
    else cout<< "IWillNeverCheatAnymore";
    return 0;
}
\end{lstlisting}
\begin{multicols}{4}
\begin{enumerate}
    \item Cheating is Bad
    \item Nothing
    \item Don't Cheat 
    \item IWillNeverCheatAnymore 
\end{enumerate}
\end{multicols}




\item Which of the following gives us a memory address of \textbf{int array[10]}:
\begin{multicols}{4}
\begin{enumerate}
    \item array[1]
    \item array[0]
    \item array 
    \item memory(array);
\end{enumerate}
\end{multicols}



\item What will be the output of this string concatenation:
\begin{lstlisting}[language=C++]
#include <iostream>
using namespace std;
int main(){
   	string x = "10";
    int y = 20;
    string z = x  y;
    return 0;
}
\end{lstlisting}
\begin{multicols}{4}
\begin{enumerate}
    \item 10+20
    \item 30
    \item 20 
    \item It wouldn’t compile because of error 
\end{enumerate}
\end{multicols}





\item What will be the output of this code. Please, write answers in the answers section.
\begin{lstlisting}[language=C++]
#include <iostream>
using namespace std;
void swap(int a, int b){
    int c;
    c = b;
    b = a;
    a = c;
}
int main(){
    int a = 5, b = 22;
    swap(a, b);
    cout << a << " " << b;
    return 0;
}
\end{lstlisting}




\item What will be the output of this code:
\begin{lstlisting}[language=C++]
#include <iostream>
#include <vector>
#include <algorithm>
#include <iterator>
using namespace std;
int main(){
    vector<int> g1; 
    for (int i = 1; i <= 5; i += 2) 
        g1.push_back(i); 
    cout << "Size : " << g1.size(); 
    return 0;
}
\end{lstlisting}
\begin{multicols}{4}
\begin{enumerate}
    \item Size : 5
    \item Size : 3
    \item 5
    \item 3 
\end{enumerate}
\end{multicols}






\item What will be the output of this code:
\begin{lstlisting}[language=C++]
#include <iostream>
#include <set>
using namespace std;
int main(){
    set<int> a;
    for (int i = 0;i < 100;i++) {
        a.insert(i);    
    }
    a.insert(43) ;
    a.insert(32+21+3);
    return 0;
}
\end{lstlisting}
\begin{multicols}{4}
\begin{enumerate}
    \item 100
    \item 102
    \item 99
    \item empty set
\end{enumerate}
\end{multicols}






\item What will be the output of this code. Please, write answers in the answers section.
\begin{lstlisting}[language=C++]
#include <iostream> 
using namespace std; 
int main() { 
    int a = 1;
    int b = 5;

    a ^= b;
    b ^= a; 
    a ^= b;
	cout << a << " " << b;
    return 0;
}
\end{lstlisting}






\item What will be the output of this code. Please, write answers in the answers section.
\begin{lstlisting}[language=C++]
#include <iostream> 
#include <map>
using namespace std; 
int main() { 
    map<string, int> a;
    string s = "1";
    for (int i = 0;i < 10;i++){
        s += char(i);
        a[s] = i;
    }
    map<string, int> :: iterator it;
    for(it = a.begin(); it != a.end(); ++it) {
        cout << it->second;
    }
    return 0;
}
\end{lstlisting}


\item What will be the output of this code. Please, write answers in the answers section.
\begin{lstlisting}[language=C++]
#include <iostream>
using namespace std;
int main(){
    int c = 5;
    if( c > 5 && c > 0 )
        cout << c++ << endl;
    if( 1 | 0 )
        c-=-1;
    cout << c << endl;
    if(c)
        ++c;
    cout << c + 1 << endl;
    return 0;
}
\end{lstlisting}





\item How many strings will be in the output and what will be the first line of the output? Please, write answers in the answers section.
\begin{lstlisting}[language=C++]
#include <algorithm>
#include <iostream>
using namespace std;
int main(){
    string s = "asd";
    sort(s.begin(), s.end());
    do{
        cout << s << endl;
    }while(next_permutation(s.begin(), s.end()));
    return 0;
}
\end{lstlisting}






\item What will be the output of this code:
\begin{lstlisting}[language=C++]
#include <iostream>
using namespace std;
int main(){
    int a[5] = {0, 1, 6, 8, 9};
    for(int i = 0; i < 6; i++)
        cout << a[i] << " ";
    return 0;
}
\end{lstlisting}
\begin{multicols}{4}
\begin{enumerate}
    \item Yes, there is no mistakes 
    \item No, there will be compilation error 
    \item No, there will be run-time error 
    \item Yes, but there will be warnings 
\end{enumerate}
\end{multicols}







\item What will be the output of this code:
\begin{lstlisting}[language=C++]
#include <iostream>
#include <string>
using namespace std;
int main(){
    string nums;
    cin >> nums;
    int ind = 0;
    for(int i = 0; i<nums.size(); i++){
        if(nums[i]!='0'){
            nums[ind] = nums[i];
            ind++;
        }
    }
    for(int i = ind; i<nums.size(); i++){
        nums[i] = '0';
    }
    cout << nums << endl;
    return 0;
}
\end{lstlisting}
\begin{multicols}{4}
\begin{enumerate}
    \item Finds the palindrome 
    \item Finds number of 0’s 
    \item Moves 0 to the end of the string 
    \item Output 0’s 
\end{enumerate}
\end{multicols}





\item Fill the gaps. Please, write answers in the answers section.
\begin{lstlisting}[language=C++]
#include <iostream>
#include <algorithm>
#include <________> //1
#include <________> //2
using namespace std;
int main(){
    stringstream ss;
    vector <string> v;
    string s = "I will never cheat on programming exams";
    ss << s;
    string word="";
    
    while(ss __ word) //3
    {
        v.push_back(word);
        word = "";
    }
    
    reverse(________________); // 4 reverse vector v
    
    for(int i=0;i<v.size();i++)
        cout<<v[i]<<endl;
    return 0;
}
\end{lstlisting}





\item Select the incorrect statements for Set (multiple choice)
\begin{multicols}{4}
\begin{enumerate}
    \item STL container
    \item unique elements 
    \item FIFO 
    \item static 
\end{enumerate}
\end{multicols}






\item Fill the gaps. What will be the output of the code below?
\begin{lstlisting}[language=C++]
#include <iostream>
#include <set>
#include <map>
using namespace std;
 
int main(){
    map <____, ____> m;
    m['C'] = 98;  m['R'] = 100;
    m['A'] = 87;  m['Z'] = 75;
    map <____, ____>:: iterator it = m.begin();
    cout << it->first << " " << it->second;
    return 0;
}

\end{lstlisting}
\begin{multicols}{4}
\begin{enumerate}
    \item string, int, A  87
    \item int, char, R  100
    \item char, int, R 100
    \item char, int, A 87 
\end{enumerate}
\end{multicols}





\item What will the “pie” equal to after the operations below?
\begin{lstlisting}[language=C++]
    int pie = 12;
    pie = pie << 2;
    pie = pie >> 1;
    pie = pie >> 3;
\end{lstlisting}
\begin{multicols}{4}
\begin{enumerate}
    \item 5
    \item 3
    \item 24
    \item 48 
\end{enumerate}
\end{multicols}







\item What is the output of following program?
\begin{lstlisting}[language=C++]
    #include <iostream>
    using namespace std;
    int main(){
        int n = 12;
        if ( (n % 3 == 0) && (n % 4 == 0) ){
            cout << n/3 << n/4;
        }else 
        cout << n;
        return 0;
    }

\end{lstlisting}
\begin{multicols}{4}
\begin{enumerate}
    \item 3 4
    \item 12
    \item 4 3
    \item 43
\end{enumerate}
\end{multicols}







\item What is the output of following program?
\begin{lstlisting}[language=C++]
#include <iostream>
using namespace std;
int main()
{
    for (int i = 0; i <= 5; i++) {
        if (i % 2 == 0) continue;
        else cout << i << " ";
   	}
    return 0;
}
\end{lstlisting}
\begin{multicols}{4}
\begin{enumerate}
    \item 0 2 4
    \item 0 1 2 3 4 5
    \item 1 3
    \item 1 3 5
\end{enumerate}
\end{multicols}




\end{enumerate}
\textbf{
Minutes \#2 of Faculty of Information Technology meeting November 11, 2019 \\\\
Teachers: Baisakov B. M., Akshabayev A. K. \\\\ 
Dean of FIT Suliev R. N \\\\
}
\end{document} 